% !TEX program = pdflatex
% !TEX encoding = UTF-8 Unicode

% Plantilla, basada en la clase `scrbook` del paquete KOMA-script,  para la elaboración de un TFG siguiendo las directrices del la comisión del Grado en Matemáticas de la Universidad de Granada.

% Francisco Torralbo Torralbo

\documentclass[print, color]{ugrTFG}

% VERSIÓN ELECTRÓNICA PARA TABLETA
% Cambiando la opción "print" por "tablet" generaremos un pdf adaptado para leerlo en tabletas de 9 pulgadas.

% -------------------------------------------------------------------
% INFORMACIÓN DEL TFG Y EL AUTOR
% -------------------------------------------------------------------

\newcommand{\miTitulo}{Título del trabajo\xspace}
\newcommand{\miNombre}{Nombre apellidos\xspace}
\newcommand{\miGrado}{Grado en Matemáticas}
\newcommand{\miFacultad}{Facultad de Ciencias}
\newcommand{\miUniversidad}{Universidad de Granada}

% Añadir tantos tutores como sea necesario separando cada uno de ellos mediante el comando `\medskip` y una línea en blanco
\newcommand{\miTutor}{
  Nombre del tutor 1 \\ \emph{Departamento del tutor 1} 

  % Añadir tantos tutores como sea necesario. 

  \medskip
  Nombre del tutor 2 \\ \emph{Departamento del tutor 2}
}
\newcommand{\miCurso}{2023-2024\xspace}

\hypersetup{
	pdftitle={\miTitulo},
	pdfauthor={\textcopyright\ \miNombre, \miFacultad, \miUniversidad}
}

\begin{document}

\maketitle

% -------------------------------------------------------------------
% FRONTMATTER
% -------------------------------------------------------------------
\frontmatter % Desactiva la numeración de capítulos y usa numeración romana para las páginas

% !TeX root = ../libro.tex
% !TeX encoding = utf8
%
%*******************************************************
% Declaración de originalidad
%*******************************************************

\thispagestyle{empty}

\hfill\vfill

\textsc{Declaración de originalidad}\\\bigskip

D./Dña. \miNombre \\\medskip

Declaro explícitamente que el trabajo presentado como Trabajo de Fin de Grado (TFG), correspondiente al curso académico \miCurso, es original, entendida esta, en el sentido de que no ha utilizado para la elaboración del trabajo fuentes sin citarlas debidamente.
\medskip

En Granada a \today 
\begin{flushleft} 
Fdo: \miNombre 

\end{flushleft}

\vfill

\cleardoublepage
\endinput
   
% !TeX root = ../libro.tex
% !TeX encoding = utf8

%*******************************************************
% Dedication
%*******************************************************
\thispagestyle{empty}
\phantomsection 
\pdfbookmark[1]{Dedicatoria}{Dedicatoria}

\hfill
\vfill

\begin{flushright}
\itshape
Dedicatoria (opcional) \\
Ver archivo \texttt{preliminares/dedicatoria.tex}
\end{flushright}

\vfill

\cleardoublepage
\endinput
                % Opcional
% !TeX root = ../libro.tex
% !TeX encoding = utf8

%*******************************************************
% Table of Contents
%*******************************************************
\phantomsection
\pdfbookmark[0]{\contentsname}{toc}

\setcounter{tocdepth}{2} % <-- 2 includes up to subsections in the ToC
\setcounter{secnumdepth}{3} % <-- 3 numbers up to subsubsections

% \manualmark
% \markboth{\textsc{\contentsname}}{\textsc{\contentsname}}
\tableofcontents 

%*******************************************************
% List of Figures and of the Tables
%*******************************************************

    % *******************************************************
    %  List of Figures
    % *******************************************************    
    \phantomsection 
    \listoffigures

    %*******************************************************
    % List of Tables
    %*******************************************************
    \phantomsection 
    \listoftables
    
    %*******************************************************
    % List of Listings
    % The package \usepackage{listings} is needed
    %*******************************************************      
	  % \phantomsection 
    % \renewcommand{\lstlistlistingname}{Listados de código}
    % \lstlistoflistings 

\cleardoublepage
            
% !TeX root = ../tfg.tex
% !TeX encoding = utf8

%*******************************************************
% Agradecimientos
%*******************************************************

\chapter{Agradecimientos}

Agradecimientos (opcional, ver archivo \texttt{preliminares/agradecimiento.tex}).

\cleardoublepage
\endinput
            % Opcional

% !TeX root = ../libro.tex
% !TeX encoding = utf8
%
%*******************************************************
% Summary
%*******************************************************


\chapter{Summary}

An english summary of the project (around 800 and 1500 words are recommended).

\endinput
                    
% !TeX root = ../libro.tex
% !TeX encoding = utf8
%
%*******************************************************
% Introducción
%*******************************************************

% \manualmark
% \markboth{\textsc{Introducción}}{\textsc{Introducción}} 

\chapter{Introducción}

Introducción del libro

\endinput
               

% -------------------------------------------------------------------
% MAINMATTER
% -------------------------------------------------------------------
\mainmatter % activa la numeración de capítulos, resetea la numeración de las páginas y usa números arábigos

\part{Primera parte} % Dividir un TFG en partes OPCIONAL

% !TeX root = ../tfg.tex
% !TeX encoding = utf8

\chapter{Primer capítulo}\label{ch:primer-capitulo}

\section{Introducción}
Este documento es una plantilla para la elaboración de un trabajo fin de Grado siguiendo los \href{https://grados.ugr.es/matematicas/pages/infoacademica/tfg/requisitosTFG}{requisitos} de la comisión de Grado en Matemáticas de la Universidad de Granada que, a fecha de junio de 2023, son las siguientes:

\begin{itemize}
  \item La  memoria  debe  realizarse  con  un  procesador  de  texto  científico,  preferiblemente (La)TeX.
  \item La portada  debe contener  el  logo  de  la UGR,  incluir  el  título del TFG, el nombre del estudiante y especificar el grado, la facultad y el curso actual.
  \item La contraportada contendrá además el nombre del tutor o tutores.
  \item La memoria debe necesariamente incluir:
    \begin{itemize}
      \item Declaración explícita firmada en la que se asume la originalidad del trabajo, entendida en el sentido de que no ha utilizado fuentes sin citarlas debidamente. Esta declaración se puede descargar en la web del Grado.
      \item un índice detallado de capítulos y secciones,
      \item un resumen amplio en inglés del trabajo realizado (se recomienda entre 800 y 1500 palabras),
      \item una introducción en la que se describan claramente los objetivos previstos inicialmente en la propuesta de TFG, indicando si han sido o no alcanzados, los antecedentes importantes para el desarrollo, los resultados obtenidos, en su caso y las principales fuentes consultadas,
      \item una bibliografía final que incluya todas las referencias utilizadas.
    \end{itemize}
  \item Se recomienda que la extensión de la memoria sea de unas 50 páginas, sin incluir posibles apéndices.
\end{itemize}

Para generar el pdf a partir de la plantilla basta compilar el fichero \texttt{tfg.tex}. Es conveniente leer los comentarios contenidos en dicho fichero pues ayudarán a entender mejor como funciona la plantilla. 

La estructura de la plantilla es la siguiente\footnote{Los nombres de las carpetas no se han acentuado para evitar problemas en sistemas con Windows}: 
\begin{itemize}
  \item Carpeta \textbf{preliminares}: contiene los siguientes archivos
    \begin{description}
      \item[\texttt{dedicatoria.tex}] Para la dedicatoria del trabajo (opcional)
      \item[\texttt{agradecimientos.tex}] Para los agradecimientos del trabajo (opcional)
      \item[\texttt{introduccion.tex}] Para la introducción (obligatorio)
      \item[\texttt{summary.tex}] Para el resumen en inglés (obligatorio)
      \item[\texttt{tablacontenidos.tex}] Genera de forma automática la tabla de contenidos, el índice de figuras y el índice de tablas. Si bien la tabla de contenidos es conveniente incluirla, el índice de figuras y tablas es opcional. Por defecto está desactivado. Para mostrar dichos índices hay que editar este fichero y quitar el comentario a \verb+\listoffigures+ o \verb+\listoftables+ según queramos uno de los índices o los dos. En este archivo también es posible habilitar la inclusión de un índice de listados de código (si estos han sido incluidos con el paquete \texttt{listings})
  \end{description}
  El resto de archivos de dicha carpeta no es necesario editarlos pues su contenido se generará automáticamente a partir de los metadatos que agreguemos en \texttt{tfg.tex}

  \item Carpeta \textbf{capitulos}: contiene los archivos de los capítulos del TFG. Añadir tantos archivos como sean necesarios. Este capítulo es \texttt{capitulo01.tex}.

  \item Carpeta \textbf{apendices}: Para los apéndices (opcional)
  \item Carpeta \textbf{img}: Para incluir los ficheros de imagen que se usarán en el documento.
    
  \item Fichero \texttt{library.bib}: Para incluir las referencias bibliográficas en formato \texttt{bibtex}. Es útil la herramienta \href{https://www.doi2bib.org/}{doi2bib} para generar de forma automática el código bibtex de una referencia a partir de su \textsc{doi}  así como la base de datos bibliográfica \href{https://mathscinet.ams.org}{MathSciNet}. Para que una referencia aparezca en el pdf no basta con incluirla en el fichero \texttt{library.bib}, es necesario además \emph{citarla} en el documento usando el comando \verb+\cite+. Si queremos mostrar todos las referencias incluidas en el fichero \texttt{library.bib} podemos usar \verb+\cite{*}+ aunque esta opción no es la más adecuada. Se aconseja que los elementos de la bibliografía estén citados al menos una vez en el documento (y de esa forma aparecerán de forma automática en la lista de referencias).

  \item Fichero \texttt{glosario.tex}: Para incluir un glosario en el trabajo (opcional). Si no queremos incluir un glosario deberemos borrar el comando \verb+% !TeX root = ../libro.tex
% !TeX encoding = utf8

\chapter*{Glosario}
\addcontentsline{toc}{chapter}{Glosario} % Añade el glosario a la tabla de contenidos

La inclusión de un glosario es opcional.

Archivo: \texttt{glosario.tex}

\begin{description} 
  \item[$\mathbb{R}$] Conjunto de números reales.

  \item[$\mathbb{C}$] Conjunto de números complejos.

  \item[$\mathbb{Z}$] Conjunto de números enteros.
\end{description}
\endinput
+ del fichero \texttt{tfg.tex} y posteriormente borrar el fichero \texttt{glosario.tex}

   \item Fichero \texttt{tfg.tex}: El documento maestro del TFG que hay que compilar con \LaTeX\ para obtener el pdf. En dicho documento hay que cambiar la \emph{información del título del \textsc{tfg} y el autor así como los tutores}.
\end{itemize}



\section{Elementos del texto}

En esta sección presentaremos diferentes ejemplos de los elementos de texto básico. Conviene consultar el contenido de \texttt{capitulos/capitulo01.tex} para ver cómo se han incluido.

\subsection{Listas}
En \LaTeX\ tenemos disponibles los siguientes tipos de listas:

Listas enumeradas:
\begin{enumerate}
  \item item 1
  \item item 2
  \item item 3
\end{enumerate}

Listas no enumeradas
\begin{itemize}
  \item item 1
  \item item 2
  \item item 3
  \end{itemize}

Listas descriptivas
\begin{description}
  \item[termino1] descripción 1
  \item[termino2] descripción 2
\end{description}
  
\subsection{Tablas y figuras}

En la \autoref{tb:ejemplo-tabla} o la \autoref{fig:logo-ugr} podemos ver\ldots

\begin{table}[htpb]
  \centering
  \begin{tabular}{ccc} \toprule
    \multicolumn{2}{c}{Agrupados} \\ \cmidrule(r){1-2}
    cabecera & cabecera & cabecera          \\ \midrule
    elemento & elemento & elemento          \\ 
    elemento & elemento & elemento          \\ 
    elemento & elemento & elemento          \\ \bottomrule
  \end{tabular}
  \caption{Ejemplo de tabla}
  \label{tb:ejemplo-tabla}
\end{table}

\begin{figure}[htpb]
  \centering
  \includegraphics[width=0.5\textwidth]{logo-ugr}
  \caption{Logotipo de la Universidad de Granada}
  \label{fig:logo-ugr}
\end{figure}

\section{Entornos matemáticos}\label{sec:entornos-matematicos}

La plantilla tiene definidos varios entornos matemáticos cuyo nombre es el mismo omitiendo los acentos. Así, para incluir una \emph{proposición} usaríamos:

\begin{verbatim}
\begin{proposicion}
texto de la proposición
\end{proposicion} 
\end{verbatim}

Ver el código fuente del archivo \texttt{capitulo01.tex} para el resto de ejemplos.

\begin{teorema}\label{thm:teorema}
Esto es un ejemplo de teorema.
\end{teorema}

\begin{proposicion}
Ejemplo de proposición
\end{proposicion}

\begin{lema}
Ejemplo de lema
\end{lema}

\begin{corolario}
Ejemplo de corolario
\end{corolario}

\begin{definicion}
Ejemplo de definición
\end{definicion}

\begin{observacion}
Ejemplo de observación
\end{observacion}

Adicionalmente está definido el entorno \texttt{teorema*} que permite incluir un teorema sin numeración:

\begin{teorema*}[Fórmula de Gauß-Bonnet]
  Sea $S$ una superficie compacta y $K$ su curvatura de Gauß. Entonces
\begin{equation}
  \int_S K = 2\pi\chi(S)
\end{equation}
donde $\chi(S)$ es la característica de Euler de $S$.
\end{teorema*}

Las fórmulas matemáticas se escriben entre símbolos de dólar \$ si van en línea con el texto o bien usando el entorno%
\footnote{
  También es posible delimitar una ecuación mediante los comandos \texttt{$\backslash$[} y \texttt{$\backslash$]} pero éstas nunca llevarán numeración aunque añadamos una etiqueta y las referenciemos (ver \autoref{sec:referencias}).
} 
\texttt{equation} cuando queremos que se impriman centradas en una línea propia, como el siguiente ejemplo
\begin{equation}\label{eq:identidad-pitagorica}
  \cos^2 x + \sin^2 x = 1.
\end{equation}


Gracias al paquete \texttt{mathtools}, las ecuaciones escritas dentro del entorno \texttt{equation} llevarán numeración de forma automática si son referenciadas  en cualquier parte del documento (por ejemplo la identidad Pitagórica~\eqref{eq:identidad-pitagorica}, ver el código de los dos anteriores ejemplos y la \autoref{sec:referencias} para más información sobre referencias cruzadas en el documento).




\section{Referencias a elementos del texto}\label{sec:referencias}

Para las referencias a los elementos del texto (secciones, capítulos, teoremas,\ldots) se procede de la siguiente manera:
\begin{itemize}
  \item Se \emph{marca} el elemento (justo antes del mismo si se trata de un capítulo o sección o en el interior del \emph{entorno} en otro caso), mediante el comando \verb+\label{+\emph{etiqueta}\verb+}+, donde \emph{etiqueta} debe ser un identificador único.
  \item Para crear una referencia al elemento en cualquier otra parte del texto se usa el comando \verb+\ref{+\emph{etiqueta}\verb+}+ (únicamente imprime la numeración asociada a dicho elemento, por ejemplo \ref{ch:primer-capitulo} o \ref{thm:teorema}) o bien \verb+\autoref{+\emph{etiqueta}\verb+}+ (imprime la numeración del elemento así como un texto previo indicando su tipo, por ejemplo \autoref{ch:primer-capitulo} o \autoref{thm:teorema})
\end{itemize}




\section{Bibliografía e índice}

Esto es un ejemplo de texto en un capítulo. Incluye varias citas tanto a libros~\cite{Aigner2018}, artículos de investigación~\cite{Euler1985}, recursos online~\cite{EulerWiki} (páginas web), tesis~\cite{CitekeyPhdthesis}, trabajo fin de máster~\cite{CitekeyMastersthesis}, trabajo fin de grado~\cite{CiteKeyBachelorsthesis} así como artículos sin publicar (preprints) \cite{castroinfantes2022conjugate} (en estos últimos usar el campo \texttt{note} para añadir la información relevante). Ver el fichero \texttt{library.bib} para las distintas plantillas. 


\endinput

% Añadir tantos capítulos como sea necesario

\cleardoublepage\part{Segunda parte}
% !TeX root = ../tfg.tex
% !TeX encoding = utf8

\chapter{Ejemplo de capítulo}

\section{Primera sección}

Este fichero \texttt{capitulo-ejemplo.tex} es una plantilla para añadir capítulos al \textsc{tfg}. Para ello, es necesario:
\begin{itemize}
  \item Crear una copia de este fichero \texttt{capitulo-ejemplo.tex} en la carpeta \texttt{capitulos} con un nombre apropiado (p.e. \texttt{capitulo01.tex}).
  \item Añadir el comando \texttt{$\backslash$input\{capitulos/capitulo01\}} en el fichero principal \texttt{tfg.tex} donde queremos que aparezca dicho capítulo.
\end{itemize}


\endinput
%--------------------------------------------------------------------
% FIN DEL CAPÍTULO. 
%--------------------------------------------------------------------


% -------------------------------------------------------------------
% APPENDIX: Opcional
% -------------------------------------------------------------------

\appendix % Reinicia la numeración de los capítulos y usa letras para numerarlos
\pdfbookmark[-1]{Apéndices}{appendix} % Alternativamente podemos agrupar los apéndices con un nuevo \part{Apéndices}

% !TeX root = ../tfg.tex
% !TeX encoding = utf8

\chapter{Ejemplo de apéndice}\label{ap:apendice1}

Los apéndices son opcionales.

Este fichero \texttt{apendice-ejemplo.tex} es una plantilla para añadir apéndices al \textsc{tfg}. Para ello, es necesario:
\begin{itemize}
  \item Crear una copia de este fichero \texttt{apendice-ejemplo.tex} en la carpeta \texttt{apendices} con un nombre apropiado (p.e. \texttt{apendice01.tex}).
  \item Añadir el comando \texttt{$\backslash$input\{apendices/apendice01\}} en el fichero principal \texttt{tfg.tex} donde queremos que aparezca dicho apéndice (debe de ser después del comando \texttt{$\backslash$appendix}).
\end{itemize}

\endinput
%------------------------------------------------------------------------------------
% FIN DEL APÉNDICE. 
%------------------------------------------------------------------------------------

% Añadir tantos apéndices como sea necesario 

% -------------------------------------------------------------------
% GLOSARIO: Opcional
% -------------------------------------------------------------------

% !TeX root = ../libro.tex
% !TeX encoding = utf8

\chapter*{Glosario}
\addcontentsline{toc}{chapter}{Glosario} % Añade el glosario a la tabla de contenidos

La inclusión de un glosario es opcional.

Archivo: \texttt{glosario.tex}

\begin{description} 
  \item[$\mathbb{R}$] Conjunto de números reales.

  \item[$\mathbb{C}$] Conjunto de números complejos.

  \item[$\mathbb{Z}$] Conjunto de números enteros.
\end{description}
\endinput
 

% -------------------------------------------------------------------
% BACKMATTER
% -------------------------------------------------------------------

\backmatter % Desactiva la numeración de los capítulos
\pdfbookmark[-1]{Referencias}{BM-Referencias}

% BIBLIOGRAFÍA
%-------------------------------------------------------------------

\bibliographystyle{alpha-es} 
\begin{small} 
  \bibliography{library.bib}
\end{small}


\end{document}
