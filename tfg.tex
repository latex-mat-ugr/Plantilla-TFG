% !TEX program = pdflatex
% !TEX encoding = UTF-8 Unicode

% Plantilla, basada en la clase `scrbook` del paquete KOMA-script,  para la elaboración de un TFG siguiendo las directrices del la comisión del Grado en Matemáticas de la Universidad de Granada.

% Francisco Torralbo Torralbo

\documentclass[print, color]{ugrTFG}

% VERSIÓN ELECTRÓNICA PARA TABLETA
% Cambiando la opción "print" por "tablet" generaremos un pdf adaptado para leerlo en tabletas de 9 pulgadas.

% -------------------------------------------------------------------
% INFORMACIÓN DEL TFG Y EL AUTOR
% -------------------------------------------------------------------

\newcommand{\miTitulo}{Título del trabajo\xspace}
\newcommand{\miNombre}{Nombre apellidos\xspace}
\newcommand{\miGrado}{Grado en Matemáticas}
\newcommand{\miFacultad}{Facultad de Ciencias}
\newcommand{\miUniversidad}{Universidad de Granada}

% Añadir tantos tutores como sea necesario separando cada uno de ellos mediante el comando `\medskip` y una línea en blanco
\newcommand{\miTutor}{
  Nombre del tutor 1 \\ \emph{Departamento del tutor 1} 

  % Añadir tantos tutores como sea necesario. 

  \medskip
  Nombre del tutor 2 \\ \emph{Departamento del tutor 2}
}
\newcommand{\miCurso}{2023-2024\xspace}

\hypersetup{
	pdftitle={\miTitulo},
	pdfauthor={\textcopyright\ \miNombre, \miFacultad, \miUniversidad}
}

\begin{document}

\maketitle

% -------------------------------------------------------------------
% FRONTMATTER
% -------------------------------------------------------------------
\frontmatter % Desactiva la numeración de capítulos y usa numeración romana para las páginas

\input{preliminares/declaracion-originalidad}   
\input{preliminares/dedicatoria}                % Opcional
\input{preliminares/tablacontenidos}            
\input{preliminares/agradecimientos}            % Opcional

\input{preliminares/summary}                    
\input{preliminares/introduccion}               

% -------------------------------------------------------------------
% MAINMATTER
% -------------------------------------------------------------------
\mainmatter % activa la numeración de capítulos, resetea la numeración de las páginas y usa números arábigos

\part{Primera parte} % Dividir un TFG en partes OPCIONAL

% Información relevante para la elaboración del trabajo.
% !TeX root = ../tfg.tex
% !TeX encoding = utf8

\chapter{Documentación}\label{ch:primer-capitulo}

\section{Introducción}
Este documento es una plantilla para la elaboración de un trabajo fin de Grado siguiendo los \href{https://grados.ugr.es/matematicas/pages/infoacademica/tfg/requisitosTFG}{requisitos} de la comisión de Grado en Matemáticas de la Universidad de Granada que, a fecha de junio de 2023, son las siguientes:

\begin{itemize}
  \item La  memoria  debe  realizarse  con  un  procesador  de  texto  científico,  preferiblemente (La)TeX.
  \item La portada  debe contener  el  logo  de  la UGR,  incluir  el  título del TFG, el nombre del estudiante y especificar el grado, la facultad y el curso actual.
  \item La contraportada contendrá además el nombre del tutor o tutores.
  \item La memoria debe necesariamente incluir:
    \begin{itemize}
      \item Declaración explícita firmada en la que se asume la originalidad del trabajo, entendida en el sentido de que no ha utilizado fuentes sin citarlas debidamente. Esta declaración se puede descargar en la web del Grado.
      \item un índice detallado de capítulos y secciones,
      \item un resumen amplio en inglés del trabajo realizado (se recomienda entre 800 y 1500 palabras),
      \item una introducción en la que se describan claramente los objetivos previstos inicialmente en la propuesta de TFG, indicando si han sido o no alcanzados, los antecedentes importantes para el desarrollo, los resultados obtenidos, en su caso y las principales fuentes consultadas,
      \item una bibliografía final que incluya todas las referencias utilizadas.
    \end{itemize}
  \item Se recomienda que la extensión de la memoria sea de unas 50 páginas, sin incluir posibles apéndices.
\end{itemize}

Para generar el pdf a partir de la plantilla basta compilar el fichero \texttt{tfg.tex}. Es conveniente leer los comentarios contenidos en dicho fichero pues ayudarán a entender mejor como funciona la plantilla. 

La estructura de la plantilla es la siguiente\footnote{Los nombres de las carpetas no se han acentuado para evitar problemas en sistemas con Windows}: 
\begin{itemize}
  \item Carpeta \textbf{preliminares}: contiene los siguientes archivos
    \begin{description}
      \item[\texttt{dedicatoria.tex}] Para la dedicatoria del trabajo (opcional)
      \item[\texttt{agradecimientos.tex}] Para los agradecimientos del trabajo (opcional)
      \item[\texttt{introduccion.tex}] Para la introducción (obligatorio)
      \item[\texttt{summary.tex}] Para el resumen en inglés (obligatorio)
      \item[\texttt{tablacontenidos.tex}] Genera de forma automática la tabla de contenidos, el índice de figuras y el índice de tablas. Si bien la tabla de contenidos es conveniente incluirla, el índice de figuras y tablas es opcional. Por defecto está desactivado. Para mostrar dichos índices hay que editar este fichero y quitar el comentario a \verb+\listoffigures+ o \verb+\listoftables+ según queramos uno de los índices o los dos. En este archivo también es posible habilitar la inclusión de un índice de listados de código (si estos han sido incluidos con el paquete \texttt{listings})
  \end{description}
  El resto de archivos de dicha carpeta no es necesario editarlos pues su contenido se generará automáticamente a partir de los metadatos que agreguemos en \texttt{tfg.tex}

  \item Carpeta \textbf{capitulos}: contiene los archivos de los capítulos del TFG. Añadir tantos archivos como sean necesarios. Este capítulo es \texttt{capitulo01.tex}.

  \item Carpeta \textbf{apendices}: Para los apéndices (opcional)
  \item Carpeta \textbf{img}: Para incluir los ficheros de imagen que se usarán en el documento.
    
  \item Fichero \texttt{library.bib}: Para incluir las referencias bibliográficas en formato \texttt{bibtex}. Es útil la herramienta \href{https://www.doi2bib.org/}{doi2bib} para generar de forma automática el código bibtex de una referencia a partir de su \textsc{doi}  así como la base de datos bibliográfica \href{https://mathscinet.ams.org}{MathSciNet}. Para que una referencia aparezca en el pdf no basta con incluirla en el fichero \texttt{library.bib}, es necesario además \emph{citarla} en el documento usando el comando \verb+\cite+. Si queremos mostrar todos las referencias incluidas en el fichero \texttt{library.bib} podemos usar \verb+\cite{*}+ aunque esta opción no es la más adecuada. Se aconseja que los elementos de la bibliografía estén citados al menos una vez en el documento (y de esa forma aparecerán de forma automática en la lista de referencias).

  \item Fichero \texttt{glosario.tex}: Para incluir un glosario en el trabajo (opcional). Si no queremos incluir un glosario deberemos borrar el comando \verb+\input{glosario.tex}+ del fichero \texttt{tfg.tex} y posteriormente borrar el fichero \texttt{glosario.tex}

   \item Fichero \texttt{tfg.tex}: El documento maestro del TFG que hay que compilar con \LaTeX\ para obtener el pdf. En dicho documento hay que cambiar la \emph{información del título del \textsc{tfg} y el autor así como los tutores}.
\end{itemize}



\section{Elementos del texto}

En esta sección presentaremos diferentes ejemplos de los elementos de texto básico. Conviene consultar el contenido de \texttt{capitulos/capitulo01.tex} para ver cómo se han incluido.

\subsection{Listas}
En \LaTeX\ tenemos disponibles los siguientes tipos de listas:

Listas enumeradas:
\begin{enumerate}
  \item item 1
  \item item 2
  \item item 3
\end{enumerate}

Listas no enumeradas
\begin{itemize}
  \item item 1
  \item item 2
  \item item 3
  \end{itemize}

Listas descriptivas
\begin{description}
  \item[termino1] descripción 1
  \item[termino2] descripción 2
\end{description}
  
\subsection{Tablas y figuras}

En la \autoref{tb:ejemplo-tabla} o la \autoref{fig:logo-ugr} podemos ver\ldots

\begin{table}[htpb]
  \centering
  \begin{tabular}{ccc} \toprule
    \multicolumn{2}{c}{Agrupados} \\ \cmidrule(r){1-2}
    cabecera & cabecera & cabecera          \\ \midrule
    elemento & elemento & elemento          \\ 
    elemento & elemento & elemento          \\ 
    elemento & elemento & elemento          \\ \bottomrule
  \end{tabular}
  \caption{Ejemplo de tabla}
  \label{tb:ejemplo-tabla}
\end{table}

\begin{figure}[htpb]
  \centering
  \includegraphics[width=0.5\textwidth]{logo-ugr}
  \caption{Logotipo de la Universidad de Granada}
  \label{fig:logo-ugr}
\end{figure}

\section{Entornos matemáticos}\label{sec:entornos-matematicos}

La plantilla tiene definidos varios entornos matemáticos cuyo nombre es el mismo omitiendo los acentos. Así, para incluir una \emph{proposición} usaríamos:

\begin{verbatim}
\begin{proposicion}
texto de la proposición
\end{proposicion} 
\end{verbatim}

Ver el código fuente del archivo \texttt{documentacion.tex} en la carpeta \texttt{capitulos} para el resto de ejemplos.

\begin{teorema}\label{thm:teorema}
Esto es un ejemplo de teorema.
\end{teorema}

\begin{proposicion}
Ejemplo de proposición
\end{proposicion}

\begin{lema}
Ejemplo de lema
\end{lema}

\begin{corolario}
Ejemplo de corolario
\end{corolario}

\begin{definicion}
Ejemplo de definición
\end{definicion}

\begin{observacion}
Ejemplo de observación
\end{observacion}

Adicionalmente está definido el entorno \texttt{teorema*} que permite incluir un teorema sin numeración:

\begin{teorema*}[Fórmula de Gauß-Bonnet]
  Sea $S$ una superficie compacta y $K$ su curvatura de Gauß. Entonces
\begin{equation}
  \int_S K = 2\pi\chi(S)
\end{equation}
donde $\chi(S)$ es la característica de Euler de $S$.
\end{teorema*}

Las fórmulas matemáticas se escriben entre símbolos de dólar \$ si van en línea con el texto o bien usando el entorno%
\footnote{
  También es posible delimitar una ecuación mediante los comandos \texttt{$\backslash$[} y \texttt{$\backslash$]} pero éstas nunca llevarán numeración aunque añadamos una etiqueta y las referenciemos (ver \autoref{sec:referencias}).
} 
\texttt{equation} cuando queremos que se impriman centradas en una línea propia, como el siguiente ejemplo
\begin{equation}\label{eq:identidad-pitagorica}
  \cos^2 x + \sin^2 x = 1.
\end{equation}


Gracias al paquete \texttt{mathtools}, las ecuaciones escritas dentro del entorno \texttt{equation} llevarán numeración de forma automática si son referenciadas  en cualquier parte del documento (por ejemplo la identidad Pitagórica~\eqref{eq:identidad-pitagorica}, ver el código de los dos anteriores ejemplos y la \autoref{sec:referencias} para más información sobre referencias cruzadas en el documento).

\section{Listados de código}

Podemos incluir un archivo externo de código mediante el comando \texttt{lstinputlisting} especificando su nombre completo (incluyendo la extensión) y usando la opción \texttt{inputpath} para indicar la ruta hacia el fichero (siempre referida a la carpeta principal de la plantilla) así como la opción \texttt{language} para indicar el lenguaje de programación en que está escrito (esto permitirá a \LaTeX\ colorear adecuadamente el código). Además, si lo consideramos necesario, podemos indicar las líneas que queremos mostrar (ver el código fuente del \autoref{code:prime}). Consultar todas las opciones posibles en la \href{https://osl.ugr.es/CTAN/macros/latex/contrib/listings/listings.pdf}{documentación del paquete \texttt{listings}}.

\lstinputlisting[inputpath=code, language=R, linerange={11-17}, firstnumber={11}, caption={Extracto código (líneas de 11 a 17) del fichero \texttt{primeR.r}}, label={code:prime}]{primeR.r}

Alternativamente, podemos incluir el código en un entorno \texttt{lstlisting} como el \autoref{code:perceptron}

\begin{lstlisting}[caption={Implementación de un perceptrón}, label={code:perceptron}, language={python}]
def dot(v, w):
    """Producto escalar de v y w, |$\color{comment}v_0 \cdot   w_0 + \cdots + v_n \cdot w_n$|"""
    return sum(v_i * w_i for v_i, w_i in zip(v, w))

def funcion_activacion(x):
    """1 si la entrada es mayor o igual que 1, 0 en otro caso."""
    return 1 if x >= 0 else 0

def perceptron(entrada, pesos):
    """1 si el perceptron se activa, 0 en otro caso"""
    return funcion_activacion(dot(entrada, pesos))
\end{lstlisting}

La opción \texttt{float} al incluir un listado de código permitará a dicho bloque ``flotar'' como si fuese un entorno \texttt{figure} y de esta manera evitaremos que se corte al final de una página.



\section{Referencias a elementos del texto}\label{sec:referencias}

Para las referencias a los elementos del texto (secciones, capítulos, teoremas,\ldots) se procede de la siguiente manera:
\begin{itemize}
  \item Se \emph{marca} el elemento (justo después del mismo si se trata de un capítulo o sección o en el interior del \emph{entorno} en otro caso), mediante el comando \verb+\label{+\emph{etiqueta}\verb+}+, donde \emph{etiqueta} debe ser un identificador único.
  \item Para crear una referencia al elemento en cualquier otra parte del texto se usa el comando \verb+\ref{+\emph{etiqueta}\verb+}+ (únicamente imprime la numeración asociada a dicho elemento, por ejemplo \ref{ch:primer-capitulo} o \ref{thm:teorema}) o bien \verb+\autoref{+\emph{etiqueta}\verb+}+ (imprime la numeración del elemento así como un texto previo indicando su tipo, por ejemplo \autoref{ch:primer-capitulo} o \autoref{thm:teorema})
\end{itemize}




\section{Bibliografía e índice}

Esto es un ejemplo de texto en un capítulo. Incluye varias citas tanto a libros~\cite{Aigner2018}, artículos de investigación~\cite{Euler1985}, recursos online~\cite{EulerWiki} (páginas web), tesis~\cite{CitekeyPhdthesis}, trabajo fin de máster~\cite{CitekeyMastersthesis}, trabajo fin de grado~\cite{CiteKeyBachelorsthesis} así como artículos sin publicar (preprints) \cite{castroinfantes2022conjugate} (en estos últimos usar el campo \texttt{note} para añadir la información relevante). 

Ver el fichero \texttt{library.bib} para las distintas plantillas. Cada nueva referencia debe añadirse en dicho fichero siguiendo el estilo del código \texttt{bibtex} según el tipo de referencia (página web, tesis, trabajo fin de grado o máster, artículo de investigación, libro,\ldots). Alternativamente se puede usar la web \href{https://zbib.org}{https://zbib.org} para generar automáticamente el código \texttt{bibtex}.


\endinput

\chapter[Consideraciones elaboración TFG]{Consideraciones generales para la elaboración de un trabajo fin de grado}

\section{Normativa de la comisión del Grado en Matemáticas}

El \textsc{tfg} lo rigen dos normativas: 
\begin{itemize}
  \item una a nivel general de la UGR (\href{https://secretariageneral.ugr.es/sites/webugr/secretariageneral/public/inline-files/BOUGR/187/PLANTILLA%20CABECERASDoc2.pdf}{Reglamento del Trabajo o Proyecto fin de Grado de la Universidad de Granada}\footnote{\url{https://secretariageneral.ugr.es/sites/webugr/secretariageneral/public/inline-files/BOUGR/187/PLANTILLA\%20CABECERASDoc2.pdf}}) y 
  \item otra complementaria a nivel de la Facultad de Ciencias  (\href{https://fciencias.ugr.es/images/stories/documentos/reglamentos/reglamentoTfgCiencias23.pdf}{Reglamento del trabajo fin de grado en la Facultad de Ciencias de la Universidad de Granada}\footnote{\url{https://fciencias.ugr.es/images/stories/documentos/reglamentos/reglamentoTfgCiencias23.pdf}}). 
\end{itemize} 
Además, la comisión del Grado de Matemáticas impone unos \href{https://grados.ugr.es/matematicas/pages/infoacademica/tfg/requisitosTFG}{Requisitos de la memoria}\footnote{\url{https://grados.ugr.es/matematicas/pages/infoacademica/tfg/requisitosTFG}}.

El \textsc{tfg} hay que elaborarlo preferiblemente en LaTeX y puede usar la plantilla disponible en \href{https://github.com/latex-mat-ugr/Plantilla-TFG/archive/master.zip}{Plantilla \textsc{tfg} grado en matemáticas formato .tex}\footnote{\url{https://github.com/latex-mat-ugr/Plantilla-TFG/archive/master.zip}}.

Toda la información anterior puede encontrarse en la \href{https://grados.ugr.es/matematicas/pages/infoacademica/trabajofingrado}{web del Grado en Matemáticas}\footnote{\url{https://grados.ugr.es/matematicas/pages/infoacademica/trabajofingrado}}.

Es conveniente tener presente la documentación anterior para la elaboración del \textsc{tfg}. En especial en lo relativo a las fechas de depósito del \textsc{tfg} para su defensa.

A continuación destaco algunos aspectos importantes de la misma:
\begin{itemize}
\item El plagio, entendido como la presentación de un trabajo u obra hecho por otra persona como propio o la copia de textos sin citar su procedencia y dándolos como de elaboración propia, conllevará automáticamente la calificación numérica de cero. Esta consecuencia debe entenderse sin perjuicio de las responsabilidades disciplinarias en las que pudieran incurrir los estudiantes que plagien.
  \item Las memorias entregadas por parte de los estudiantes tendrán que ir firmadas sobre una declaración explícita en la que se asume la originalidad del trabajo, entendida en el sentido de que no ha utilizado fuentes sin citarlas debidamente.
\end{itemize}


% Los criterios de evaluación utilizados permitirán evaluar el grado de adquisición de las competencias que tiene establecidas el TFG en el VERIFICA de la titulación. Además, entre otros aspectos, se tendrá en consideración:

% \begin{itemize}
%   \item Redacción y ortografía tanto en la memoria del TFG como en los medios usados para la defensa del mismo (diapositivas, etc.).
%   \item Adecuación al formato de memoria indicado. Se proporcionará una plantilla de memoria de TFG a tal fin.
%   \item Adecuación temporal a los cronogramas de trabajo según los plazos de entrega marcados por el tutor/es.
%   \item Nivel de profundidad en los contenidos expuestos.
%   \item Dominio del tema e iniciativa del alumno.
%   \item Claridad de la exposición y adecuación al tiempo de exposición establecido.
%   \item Capacidad de análisis y síntesis.
%   \item Discusión con la Comisión Evaluadora en el turno de preguntas.
% \end{itemize}

% Se entregará una copia escrita, a doble cara, de la memoria del TFG para su evaluación, por parte de la Comisión Evaluadora, con una antelación de una semana (7 días naturales) antes de la fecha de defensa pública del TFG. Además, se entregará una versión en formato "pdf" de dicha memoria que quedará en la base de datos de todos los TFG y que custodiará la CTFGOO.




\section{Formato de la memoria}
La memoria se presentará usando un editor de textos científico, preferiblemente \LaTeX, e incluir los siguientes apartados:
\begin{enumerate}
  \item \emph{Resumen en inglés}: Deberá estar escrito completamente en inglés y tener una longitud recomendada entre 800 y 1500 palabras. 
  \item \emph{Introducción}. Deberá:
    \begin{itemize}
      \item Indicar los \emph{Objetivos del trabajo}: deberán aparecer con claridad los objetivos inicialmente previstos en la propuesta de \textsc{tfg} y los finalmente alcanzados con indicación de dificultades, cambios y mejoras respecto a la propuesta inicial. Si procede, es conveniente apuntar de manera precisa las interdependencias entre los distintos objetivos y conectarlos con los diferentes apartados de la memoria. Se pueden destacar aquí los aspectos formativos previos más utilizados. 
    \item Contextualizar el trabajo explicando antecedentes importantes para el desarrollo realizado y efectuando, en su caso, un estudio de los progresos recientes.
    \item Describir el problema abordado, de forma que el lector tenga desde este momento una idea clara de la cuestión a resolver o del producto a desarrollar y una visión general de la solución alcanzada.
    \item Indicar los resultados obtenidos.
    \item Citar las principales fuentes consultadas.
    \end{itemize}

  \item \emph{Desarrollo del trabajo}: El trabajo se estructurará en partes o capítulos según convengan, con la posibilidad de incluir apéndices. Se recomienda que la extensión de esta parte (sin incluir los apéndices) sea de unas 50 páginas. 

  \item \emph{Conclusiones y vías futuras}: Las conclusiones deberán incluir todas aquellas de tipo profesional y académico. Si hubiese posibles vías claras de desarrollo posterior sería interesante destacarlas aquí, poniéndolas en valor en el contexto inicial del trabajo.

  \item \emph{Bibliografía final}: Se incluirán tanto las fuentes primarias como todas aquellas cuyo peso haya sido menor en la realización del trabajo. Se recomienda un breve comentario de las referencias, ya sea individualizado, por grupos de referencias o global. En caso de incluir \textsc{url}s de páginas web deberán ir acompañadas de título, autor y fecha de último acceso, entre otros datos relevantes. Se recomienda no abusar de este tipo de fuentes.
\end{enumerate}




\section{Recomendaciones}
A la hora de abordar un trabajo como este, de cierta complejidad y extensión, es conveniente tener ciertas consideraciones desde un principio que ayuden a la organización y realización del mismo.
\begin{itemize}
    \item La memoria deberá ceñirse a las directrices dadas en la sección precedente. 
    
    \item Cualquier consulta externa (libro, artículo, página web, imagen,\ldots) debe estar debidamente referenciada tanto en el texto como en la bibliografía al final del trabajo. La bibliografía debe de aparecer en orden alfabético (del primer autor) en el formato indicado en la plantilla. 

    \item Se debe evitar copiar texto de forma literal, salvo citas literales, que se indicarán como tales y entrecomilladas. LaTeX proporciona el entorno \texttt{quote} para ello.

    \item Todas las imágenes y tablas incluidas en el documento deben figurar con su respectivos créditos (excepto que sean de elaboración propia). Por tanto, es recomendable guardar las referencias consultadas (direcciones web, libros) para la obtención de cualquier material gráfico o de datos.

    \item Si el trabajo contiene gran cantidad de vocabulario específico, conviene añadir un glosario de términos al final del mismo. Esto es mejor ir haciéndolo conforme se avanza en la redacción del trabajo.

    \item Es conveniente hacer un esquema inicial con la estructura general de la memoria: ¿de cuántas partes constará? ¿en qué orden? ¿qué incluirá cada una de ellas? En la plantilla proporcionada se recomienda una estructura general. Ello ayudará a organizar mejor el trabajo. No obstante, dicha estructura inicial puede ser modificada cuando el trabajo esté avanzado si el contenido lo requiere.
\end{itemize}


% Añadir tantos capítulos como sea necesario

\cleardoublepage\part{Segunda parte}
\input{capitulos/capitulo-ejemplo}

% -------------------------------------------------------------------
% APPENDIX: Opcional
% -------------------------------------------------------------------

\appendix % Reinicia la numeración de los capítulos y usa letras para numerarlos
\pdfbookmark[-1]{Apéndices}{appendix} % Alternativamente podemos agrupar los apéndices con un nuevo \part{Apéndices}

\input{apendices/apendice-ejemplo}
% Añadir tantos apéndices como sea necesario 

% -------------------------------------------------------------------
% GLOSARIO: Opcional
% -------------------------------------------------------------------

\input{glosario} 

% -------------------------------------------------------------------
% BACKMATTER
% -------------------------------------------------------------------

\backmatter % Desactiva la numeración de los capítulos
\pdfbookmark[-1]{Referencias}{BM-Referencias}

% BIBLIOGRAFÍA
%-------------------------------------------------------------------

\bibliographystyle{alpha-es} 
\begin{small} 
  \bibliography{library.bib}
\end{small}


\end{document}
